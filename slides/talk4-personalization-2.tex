% to copy from the backup section of federated_learning_optim.tex
\begin{document}

%----------------------------------------------------------------------------------------
%	TITLE PAGE
%----------------------------------------------------------------------------------------

\title[Personalization]{Talk 4: Personalization in Federated Learning --- II}
\date{2021年6月10日}
\author[]{WEN Hao}

% \institute[北京航空航天大学] % Your institution as it will appear on the bottom of every slide, may be shorthand to save space
% {
% 数学科学学院 \\ % Your institution for the title page
% \medskip
% \textit{wenh06@gmail.com} % Your email address
% 北京航空航天大学 \\
% 数学科学学院 \qquad 北京航空航天大学
% }

% \logo{\includegraphics[height=1.5cm]{logo}}
% \logoii{\includegraphics[height=1cm]{logo2}}

% \date{\footnotesize 2021年4月13日} % Date, can be changed to a custom date

\setlength{\belowdisplayskip}{5pt} \setlength{\belowdisplayshortskip}{5pt}
\setlength{\abovedisplayskip}{5pt} \setlength{\abovedisplayshortskip}{5pt}

%------------------------------------------------

\begin{frame}
\titlepage % Print the title page as the first slide
\end{frame}

\begin{frame}
\frametitle{Personalization for FL}

{\bfseries When does one need personalization?}

\vspace{0.2em}
\noindent --- When data across clients are ``enough'' non-IID, which is more realistic.

\pause
\vspace{0.8em}

{\bfseries Means of personalization:}
\begin{itemize}
    \item Federated Multi-Task Learning (+ regularization / proximal term), e.g. \cite{smith2017mocha}
    \item Model-Agnostic Meta Learning, e.g. \cite{finn2017maml}
    \item Local Fine-tuning.
    \item etc.
\end{itemize}

\end{frame}

%------------------------------------------------

\section[MAML]{Model-Agnostic Meta Learning}

%------------------------------------------------
% Page 15

\begin{frame}
\frametitle{Model-Agnostic Meta Learning}

\begin{quote}
    ``The goal of meta-learning is to train a model on a {\bfseries variety of learning tasks}, such that it can solve new learning tasks using only a small number of training samples.'' \hfill -- \cite{finn2017maml}
\end{quote}

\vspace{0.6em}

a distribution of learning tasks $p(\mathcal{T})$,

\end{frame}

%------------------------------------------------

\section[FMTL]{Federated Multi-Task Learning}

%------------------------------------------------
% Page 15

\begin{frame}
\frametitle{Federated Multi-Task Learning}

\begin{itemize}
    \pgfsetfillopacity{0.4}
    \item Mixture of global and local \cite{hanzely2020federated}:
    $$\text{minimize} \quad \sum\limits_{i=1}^N f_i(x_i) + \dfrac{\lambda}{2} \sum\limits_{i=1}^N \lVert x_i - {\color{red} \overline{x}} \rVert^2$$
    \pgfsetfillopacity{1}
    \item pFedMe (bi-level) \cite{t2020pfedme} (and similarly EASGD\cite{zhang2015easgd}):
    \begin{align*}
        & \text{minimize} \quad \sum\limits_{i=1}^N F_i(x), \\
        & \text{where} \quad F_i(x) = \min \left\{ f_i(x_i) + \dfrac{\lambda}{2} \lVert x_i - {\color{red} x}\rVert^2 \right\}
    \end{align*}
    \item FedU \cite{dinh2021fedu}:
    $$\text{minimize} \quad \sum\limits_{i=1}^N f_i(x_i) + \dfrac{\lambda}{2} \sum\limits_{i=1}^N {\color{red} \sum\limits_{j\in\mathcal{N}_i}} \lVert x_i - {\color{red} x_j} \rVert^2$$
\end{itemize}

\end{frame}

%------------------------------------------------
% Page 15

\begin{frame}

to update....

\end{frame}

%------------------------------------------------
% Page 15

\begin{frame}

to update....

\end{frame}

%------------------------------------------------
% Page 15

\begin{frame}

to update....

\end{frame}

%------------------------------------------------
% Page 15

\begin{frame}

to update....

\end{frame}

%------------------------------------------------
% Page 15

\begin{frame}

to update....

\end{frame}

%------------------------------------------------
% Page 15

\begin{frame}

to update....

\end{frame}

%------------------------------------------------
% Page 15

\begin{frame}

to update....

\end{frame}

%------------------------------------------------
% Page 15

\begin{frame}[allowframebreaks]
\frametitle{参考文献}

{\footnotesize
\bibliographystyle{ieeetr}
\bibliography{references}
}

\end{frame}

%------------------------------------------------

\end{document}